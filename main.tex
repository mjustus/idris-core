\documentclass{article}

\usepackage{amsmath}
\usepackage{amssymb}

\usepackage{mathtools}

\usepackage{inputenc}
\usepackage{fontenc}

\usepackage{stmaryrd}

\usepackage{mathpartir}
\renewcommand{\TirNameStyle}[1]{{\scriptsize {\textsc {[#1]}}}}

\usepackage{url}
\usepackage{hyperref}

\usepackage{xcolor}

\definecolor{term}{RGB}{153,0,0}
\definecolor{type}{RGB}{0,0,224}
\definecolor{definition}{RGB}{0,102,0}
\definecolor{neutral}{RGB}{102,0,102}
\definecolor{meta}{RGB}{224,153,0}

\usepackage{sourcecodepro}

% https://raw.githubusercontent.com/idris-hackers/idris-extras/master/idrislang.sty
\usepackage{idrislang}

\usepackage{todonotes}

%\usepackage[top=0.75in, bottom=1in, left=1in, right=1in]{geometry}

\title{Yaffle}

\newcommand{\used}[2]{\textsf{used}(#1,#2)}
\newcommand{\usageCount}[2]{\textsf{count}(#1,#2)}

\newcommand{\ceil}[1]{\left\lceil #1\right\rceil}
%\newcommand{\restrictOne}[1]{\left[ #1\right]_{\mathbb I}}
%\newcommand{\presence}[1]{\left\vert#1\right\vert_{\zero\one}}
\newcommand{\presence}[1]{\left\vert#1\right\vert}

\newcommand{\delete}[1]{\left\lvert #1\right\rvert}
\newcommand{\deleteQ}[2]{\textsf{erase}^{#2}\left(#1\right)}

\newcommand{\USES}{\triangleright}
\newcommand{\ENT}{\vdash}
\newcommand{\ENTq}[1]{\vdash^{#1}}
\newcommand{\OF}{:}
\newcommand{\TO}{\Rightarrow}
\newcommand{\INq}[1]{}
\newcommand{\OFq}[1]{\OF^{#1}}
\newcommand{\TOq}[1]{\TO^{#1}}
\newcommand{\EQ}{\ensuremath{\mathbin{\equiv}}}

\newcommand{\EMPTY}{.}

\newcommand{\checkJ}[5]{#1; #2 \ENT {#3} \OFq {#4} #5}

\newcommand{\checkPatJ}[5]{#1; #2; #3 \ENT {#4} \OF #5}
\newcommand{\matchJ}[3]{\textsf{Match}(#1, #2, #3)}

\newcommand{\convJ}[5]{\Sigma; #1 \ENT {#2} = {#3} \OFq {#4} #5}

\newcommand{\linCheckJ}[4]{\Sigma; #1 \ENT {#2} \USES^{#3} {#4}}

\newcommand{\eqJ}[6]{#1; #2 \ENT {#3} = {#4} \OFq {#5} #6}
\newcommand{\checkEqJ}[6]{#1; #2 \ENT {#3} = {#4} \OFq {#5} #6}

\newcommand{\type}[1]{{\textcolor{type}{\textsf{#1}}}}
\newcommand{\term}[1]{{\textcolor{term}{\textsf{#1}}}}
\newcommand{\definition}[1]{{\textcolor{definition}{\textsf{#1}}}}
\newcommand{\unv}[1]{{\textcolor{type}{\textsc{#1}}}}

\newcommand{\zero}{{\textcolor{type}{0}}}
\newcommand{\one}{{\textcolor{type}{1}}}
\newcommand{\any}{{\textcolor{type}{\omega}}}

\newcommand{\restrictQ}[2]{#1 \setminus #2}
\newcommand{\multQ}[2]{{#1} \cdot {#2}}
\newcommand{\multQE}[2]{{#1} {#2}}
\newcommand{\restrictAny}[1]{\restrictQ {#1} \any}

\newcommand{\name}[1]{{\textcolor{neutral}{#1}}}
\newcommand{\metaVariable}[1]{{\textcolor{meta}{#1}}}

\newcommand{\Type}{\unv{Type}}
\newcommand{\One}{\type{1}}
\newcommand{\To}{\mathbin{\textcolor{type}{\shortrightarrow}}}
% \newcommand{\App}[3]{#2^{#1}\, #3}
\newcommand{\App}[3]{\term{App}^{#1}\left(#2,#3\right)}
\newcommand{\Sub}[3]{#1 [#2/{\name #3}]}
\newcommand{\SubE}[2]{#1 [#2]}
\newcommand{\AppTele}[2]{\overline{\term{App}}\left(#1, #2\right)}
\newcommand{\MetaApp}[3]{\term{MetaApp}\left(\metaVariable {#1} \overline{,^{#2} {#3}}\right)}
\newcommand{\TyCon}[1]{\type{TyCon}\left({\definition {#1}}\right)}
\newcommand{\DataCon}[1]{\term{DataCon}\left({\definition {#1}}\right)}
\newcommand{\lam}[3]{\textcolor{term}{\lambda} {(\name {#1} \OFq {#2} {#3})}.\,}
\newcommand{\Let}[5]{\term{let}\, {\name #1} \OFq {#2} {#3} := {#4} \mathbin{\term{in}} {#5}}

% \newcommand{\Inf}[1]{\overline{#1}}
% \newcommand{\Annot}[2]{\left(#1 \OF #2\right)}

\newcommand{\fv}[1]{\text{fv}(#1)}

\newcommand{\Def}[2]{\definition{#1} := {#2}}
\newcommand{\Sig}[3]{\definition{#1} \OFq {#2} {#3}}
\newcommand{\checkDefJ}[1]{{} \ENT {#1}}

\newcommand{\Case}[4]{\term{case}~ {#1} \OFq {#2} {#3} \mathrel{\term{of}} \left\{\overline{#4}\right\}}

\newcommand{\Clause}[2]{{#1} \shortrightarrow {#2}}

\newcommand{\mult}{\mathbin{\textcolor{type}{\times}}}
\newcommand{\proj}[1]{\mathop{\textcolor{definition}{\pi_{#1}}}}
\newcommand{\cons}[2]{\textcolor{term}{[}#1\textcolor{term}{,}\, #2\textcolor{term}{]}}
\newcommand{\nil}{\textcolor{term}{[]}}

\newcommand{\SigmaT}[3]{\textcolor{type}{(}\name #1 \OF #2 \textcolor{type}{)} \mult #3}
\newcommand{\PiT}[4]{\textcolor{type}{(}\name #1 \OF^{#2} #3 \textcolor{type}{)} \To #4}

% \newcommand{\PiT}[4]{\type{\Pi}\textcolor{type}{(}\name #1 \OF^{#2} #3 \textcolor{type}{)}. #4}
\newcommand{\Fun}{\type{\Pi}}
\newcommand{\PiTs}[5]{\underset{#1}{\Fun}\textcolor{type}{(}\name #2 \OF^{#3} #4 \textcolor{type}{)}. #5}

\newcommand{\PiTele}[4]{\overline{\textcolor{type}{(}\name #1 \OF^{#2} #3 \textcolor{type}{)}} \To #4}

\newcommand{\plus}{\mathbin{\textcolor{type}{+}}}
\newcommand{\inj}[1]{\mathop{\textcolor{term}{\iota_{#1}}}}
\newcommand{\match}[2]{\textcolor{definition}{\langle}#1\textcolor{definition}{,}\, #2\textcolor{definition}{\rangle}}

%%%%%%%%%%%%%%%%%%%%%%%%%%%%%%%%%%%%%%%%%%%%%%%%%%%%%%%%%%%%%%%%%%%%%%%%%%%%%%%%%%%%
% \newcommand{\cOne}{\con \App \cons \dOne \nil}                                   %
% \newcommand{\cSigma}[2]{\con \App \cons \dSigma {\cons {#1}  {\cons {#2} \nil}}} %
% \newcommand{\cInd}[1]{\con \App \cons \dInd {\cons #1 \nil}}                     %
% \newcommand{\cHind}[2]{\con \App \cons \dHind {\cons #1  {\cons #2 \nil}}}       %
%                                                                                  %
% \newcommand{\uOne}{\underline{\dOne}}                                            %
% \newcommand{\uSigma}[2]{\underline{\dSigma} \App #1 \App #2}                     %
% \newcommand{\uInd}[1]{\underline{\dInd} \App #1}                                 %
% \newcommand{\uHind}[2]{\underline{\dHind} \App #1 \App #2}                       %
%                                                                                  %
% \newcommand{\con}{\term{con}}                                                    %
%%%%%%%%%%%%%%%%%%%%%%%%%%%%%%%%%%%%%%%%%%%%%%%%%%%%%%%%%%%%%%%%%%%%%%%%%%%%%%%%%%%%

\usepackage{biblatex}
\bibliography{references.bib}

\begin{document}

\maketitle

We extract a set of typing rules for the hole-free fragment of Idris' new core, Yaffle\footnote{See \href{https://github.com/edwinb/Yaffle/blob/main/src/Core/Check/Typecheck.idr}{{\texttt{Core.Check.Typecheck}}} and \href{https://github.com/edwinb/Yaffle/blob/main/src/Core/Check/Linear.idr}{\texttt{Core.Check.Linear}} in repository \url{https://github.com/edwinb/Yaffle}.}

Idris splits the world into expressions that are present at runtime and those that may be erased after type-checking. This is enforced via quantity annotations on variables in context
\begin{itemize}
\item quantities $p, q \mathrel{\text{::=}} \zero, \one, \any$ with $\zero < \one < \any$
\item contexts $\Gamma \mathrel{\text{::=}} \EMPTY \mid \Gamma, (\name x \OFq q T)$
\end{itemize}
Informally, runtime-present expressions can only depend on variables at quantity $\one$ or $\any$, where $\one$-annotated variables must be used exactly once and $\any$-annotated variables can be used arbitrarily. In contrast, erased expressions have unrestricted access to all variables, including $\zero$-annotated variables.

The context records the number of copies available of each variable; it does not account for exact variable usage inside the term to be checked\footnote{This is departure from Atkey's presentation of QTT\cite{Atkey2018}.}. Type checking only enforces that variable usage adheres to this budget locally but, in particular, does not enforce that linear variables are used precisely once. This is left to the \hyperref[sec:linearity_checking]{linearity checker}, which runs after type checking.

We will frequently need to multiply quantities, turning the set of  quantities into a monoid
\begin{align*}
\multQ \one y &= y\\
\multQ x \one &= x\\\\
\multQ \zero y &= \zero\\
\multQ x \zero &= \zero\\\\
\multQ \any \any &= \any
\end{align*}

When manipulating the context, we will also need a restriction operation $\restrictQ p q$, which makes $p$-annotated variables that do not support at least $q$ copies unavailable at runtime
$$
\restrictQ p q =
  \left\{
    \begin{array}{ll}
      p & \text{if $q \leq p$} \\
      \zero & \text{otherwise}
    \end{array}
  \right.
$$
Concretely,
\begin{align*}
\restrictQ p \zero &= p\\
\restrictQ p \one &= p\\
\restrictAny \zero &= \zero\\
\restrictAny \one &= \zero\\
\restrictAny \any &= \any
\end{align*}
and lifted to contexts
\begin{align*}
\restrictQ {\EMPTY} p &= \EMPTY\\
\restrictQ {\Gamma, (x \OFq q T)} p &= \restrictQ \Gamma p, (x \OFq {\restrictQ q p} T)
\end{align*}
Note that $\Gamma = \restrictQ \Gamma \zero = \restrictQ \Gamma \one$.

\section{Quantity-annotated judgements}

The type-checker follows a bidirectional discipline, elaborating a domain-free\footnote{Meaning binders are not type-annotated, c.f.~\cite{Barthe2000}} bidirectional language\footnote{Checkable terms \texttt{RawC} and inferrable terms \texttt{RawI} are defined in \href{https://github.com/edwinb/Yaffle/blob/main/src/Core/Syntax/Raw.idr}{Core.Syntax.Raw}.} to a conventional domain-full language\footnote{Defined as \texttt{Term} in \href{https://github.com/edwinb/Yaffle/blob/main/src/Core/TT/TT.idr}{Core.TT.TT}.}. We give a type system for the latter.

Judgement $\checkJ \Sigma \Gamma t p A$ amounts to checking that term $t$ has type $A$ in context $\Gamma$ at {\em ambient} quantity $p$ with definitions $\Sigma$. The ambient quantity has dual purpose. Much like QTT's erased/present flag~\cite{Atkey2018}, it controls whether the expression to be checked is considered to be erased ($\zero$) or present ($\one$ or $\any$). It also acts as cost multiplier for variable accesses: suppose we want to infer the type of an application $\App \any f t$ at ambient quantity $\one$ and already know that $\checkJ \Sigma \Gamma f \one {\PiT x \any A B}$. We now not only need to check that argument $t$ has type $A$ but also that we are allowed to make $\any$ copies of $t$. One way to achieve this would be to restrict the context to those variables that support $\any$ copies, i.e. $\checkJ \Sigma {\restrictAny \Gamma} t {\one} A$. Idris, however, uses a variable rule that requires any accessed variable to have quantity larger or equal to the ambient quantity
\begin{mathpar}
\inferrule*[Right=Var]{
  (\name x \OFq q A) \in \Gamma\\
  p \leq q
} {
  \checkJ \Sigma \Gamma {\name x} p A
}
\end{mathpar}
allowing us to use $\checkJ \Sigma \Gamma t \any A$ instead. In fact, we should always have $$\checkJ \Sigma \Gamma t {p} A \iff \checkJ \Sigma {\restrictQ \Gamma p} t {\presence p} A$$ where $\presence p$ squashes quantity $p$ down to just its presence
\begin{mathpar}
\begin{array}{l}
\presence \zero = \zero\\
\presence \one = \one\\
\presence \any = \one
\end{array}
\end{mathpar}

\section{Syntax}

$$
\begin{array}{lrll}
  \text{expressions } {t, s, e, f, A, B} &\mathrel{\text{::=}} &\name x & \text{variables ${\name x} \in \mathbb V$}\\
  &\mid &\definition f & \text{defined functions}\\
  &\mid &\TyCon {\definition tc} & \text{type constructors}\\
  &\mid &\DataCon {\definition dc} & \text{data constructors}\\
  &\mid &\lam x q A t\\
  &\mid &\Let x q A t s\\
  &\mid &\PiT x q A B & \text{function types}\\
  &\mid &\App q f t & \text{function application}\\
  &\mid &\Case t q A {\Clause l {t_i}} & \text{case split}\\
  &\mid &c & \text{constants}\\
  &\mid &\Type\\
  \\
  \text{constants } {c} &\mathrel{\text{::=}} &\term{n} & \text{$\term{n} \in \mathbb N$}\\
  &\mid &\ldots\\
  &\mid &\type{Int}\\
  &\mid &\ldots\\
  \\
  \text{patterns } {l} &\mathrel{\text{::=}} &\definition{dc}~ \Phi \\
  &\mid &\term{\_} & \text{catch-all pattern} \\
  &\mid &\term{n}\\
  &\mid &\ldots\\
  \\
  \text{declarations } {d} &\mathrel{\text{::=}} &\definition f \OFq q A & \text{function signature}\\
  &\mid &\definition f = t & \text{function definition}\\
  &\mid &\definition{tc} \OF \Delta \to \Type & \text{type constructor}\\
  &\mid &\definition{dc} \OF \Phi \to \AppTele {\TyCon {\definition {tc}}} \delta  & \text{data constructor}
\end{array}
$$

\subsection{Derived syntax}

We define telescopes
$$
\Delta, \Phi \mathrel{\text{::=}} \EMPTY \mid (\name x \OFq q A), \Delta
$$
and environments
$$
\delta, \phi \mathrel{\text{::=}} \EMPTY \mid t^q, \delta
$$
as a syntactic convenience for defining data types. These give rise to telescope abstraction $\Delta \to A$
\begin{align*}
  {\EMPTY \to A} &= A\\
  {(\name x \OFq q A), \Delta \to A} &= \PiT x q A {(\Delta \to A)}
\end{align*}
and environment application $\AppTele f \delta$
\begin{align*}
  {\AppTele f \EMPTY} &= f\\
  {\AppTele f {(t^q, \delta)}} &= \AppTele {\App q f t} \delta
\end{align*}

\section{Type checking}

\subsection{Type conversion}
\begin{mathpar}
\inferrule*[Right=Conv]{
  \checkJ \Sigma \Gamma e p B\\
  \convJ \Gamma A B \zero \Type
} {
  \checkJ \Sigma \Gamma e p A
}
\end{mathpar}

\subsection{Variables}

\begin{mathpar}
\inferrule*[Right=Var]{
  (\name x \OFq q A) \in \Gamma\\
  p \leq q
} {
  \checkJ \Sigma \Gamma {\name x} p A
}
\end{mathpar}

\subsection{Function type}

\begin{mathpar}
\inferrule*[Right=Pi]{
  \checkJ \Sigma \Gamma A \zero \Type \\
  \checkJ \Sigma {\Gamma, (\name x \OFq q {A})} B \zero \Type
} {
  \checkJ \Sigma \Gamma {\PiT x q A B} p \Type
}
\end{mathpar}

\subsection{Function abstraction}

\begin{mathpar}
\inferrule*[Right=Lam]{
  \checkJ \Sigma {\restrictQ \Gamma p, (\name x \OFq q A)} t {\presence p} B
} {
  \checkJ \Sigma \Gamma {\lam x q A t} p {\PiT x q A B}
}
\end{mathpar}

\subsection{Function application}
\begin{mathpar}
\inferrule*[Right=App] {
  \checkJ \Sigma \Gamma f p {\PiT x q A B} \\
  \checkJ \Sigma \Gamma t {pq} A
} {
  \checkJ \Sigma \Gamma {\App q f t} p {\Sub B t x}
}
\end{mathpar}
Note: Idris only restricts the context when introducing a linear variable at ambient quantity $\any$. We do not replicate this optimisation in the rules.

\subsection{Let binding}
\begin{mathpar}
\inferrule*[Right=Let]{
  \checkJ \Sigma \Gamma A \zero \Type\\
  \checkJ \Sigma \Gamma t {\multQE p q} A\\
  \checkJ \Sigma {\restrictQ \Gamma p, (\name x \OFq q A)} s {\presence p} B
} {
  \checkJ \Sigma \Gamma {\Let x q A t s} p B
}
\end{mathpar}

\subsection{Case}
The aim is to have typing rules for case splits that go beyond the expressivity of eliminators. We would like to end up with something akin to the \textsf{with}-construct of~\cite{Mcbride2004} but with the additional information obtained algorithmicly made explicit. This will require not just a type annotation describing the motive\footnote{Similar to the \href{https://coq.github.io/doc/master/refman/language/core/inductive.html\#the-match-with-end-construction}{\texttt{return}-annotation} in Coq.} but also explicit substitutions capturing the result of unification.

Since this is still work-in-progress, we present rules that are close to what is currently implemented but without replicating the construction of the goal type.

\todo[noline]{Telescope to context conversion.}\begin{mathpar}
\inferrule*[Right=Case]{
  \checkJ \Sigma \Gamma t {\multQE p q} A \\
  \checkPatJ \Sigma \Gamma A {l_i} {\Phi_i}\\
  \checkJ \Sigma {\restrictQ \Gamma p, \multQ {\Phi_i} q} {t_i} {\presence p} {?}\\\\
  \presence p = \one \land q = \zero \implies i \leq 1
} {
  \checkJ \Sigma \Gamma {\Case t q A {\Clause {l_i} {t_i}}} q {?}
}
\end{mathpar}
Note: we currently do not check coverage.

\subsubsection{Patterns}
\begin{mathpar}
\inferrule*[Right=Pat-Default]{ } {
  \checkPatJ \Sigma \Gamma A {\term{\_}} \EMPTY
}
\end{mathpar}

\begin{mathpar}
\inferrule*[Right=Pat-Int]{
  (\term n \in \mathbb N)
} {
  \checkPatJ \Sigma \Gamma {\type{Int}} {\term{n}} \EMPTY
}
\end{mathpar}

\begin{mathpar}
\inferrule*[Right=Pat-DataCon]{
  \left( \definition{tc} \OF {\Delta \to \Type} \right) \in \Sigma\\\\
  \left( \definition{dc} \OF {\Phi \to {\AppTele {\TyCon {tc}} \delta}} \right) \in \Sigma
} {
  \checkPatJ \Sigma \Gamma {\AppTele {\definition {tc}} \delta} {\definition{dc} ~ \Phi} \Phi
}
\end{mathpar}
Note: this rule is overly restrictive/wrong; the two environments do not need to be identical, only ``compatible''.

\subsection{Constants}
\begin{mathpar}
\inferrule*[Right=Int]{ } {
  \checkJ \Sigma \Gamma {\type{Int}} p \Type
}
\end{mathpar}

\begin{mathpar}
\inferrule*[Right=Int-Const]{
  (\term {n} \in \mathbb N)
} {
  \checkJ \Sigma \Gamma {\term {n}} p {\type{Int}}
}
\end{mathpar}

\subsection{Universe}
\begin{mathpar}
\inferrule*[Right=Type]{ } {
  \checkJ \Sigma \Gamma \Type p \Type
}
\end{mathpar}
Note: Yaffle currently has type-in-type but cumulative universes are on the roadmap.

\subsection{Defined functions}
\begin{mathpar}
\inferrule*[Right=ref] {
  \left( \Sig f q A \right) \in \Sigma\\
  {\presence p} \leq q
} {
  \checkJ \Sigma \Gamma {\definition f} p A
}
\end{mathpar}

\subsection{Data types}
\begin{mathpar}
\inferrule*[Right=TyCon] {
  \left( \definition{tc} \OF {\Delta \to \Type} \right) \in \Sigma
} {
  \checkJ \Sigma \Gamma {\TyCon {tc}} p {\Delta \to \Type}
}
\end{mathpar}

\begin{mathpar}
\inferrule*[Right=DataCon] {
  \left( \definition{dc} \OF {\Phi \to {\AppTele {\TyCon {tc}} \delta}} \right) \in \Sigma
} {
  \checkJ \Sigma \Gamma {\DataCon {dc}} p {\Phi \to {\AppTele {\TyCon {tc}} \delta}}
}
\end{mathpar}


% \subsection{Meta-variables}
% \begin{mathpar}
% \inferrule*[Right=MetaApp-def] {
%   (X = d) \in \Sigma \\
%   \Gamma \ENTq {pq_i} t_i \leadsto {t'_i} \USES u_i
% } {
%   \Gamma \ENTq p {\MetaApp X {q_i} {t_i}} \leadsto \MetaApp X {q_i} {\deleteQ {t'_i} {q_i}} \USES \bigcup u_i
% }

% \inferrule*[Right=MetaApp-undef] {
%   X \in \Sigma
% } {
%   \Gamma \ENTq p {\MetaApp X {q_i} {t_i}} \leadsto \MetaApp X {q_i} {\deleteQ {t_i} {q_i}} \USES \emptyset
% }
% \end{mathpar}

\section{Definitions}

A Yaffle file consists of a sequence of declarations
$$\Sigma \mathrel{\text{::=}} \EMPTY \mid {\Sigma, d}$$
which we call well-typed when judgement $\checkDefJ \Sigma$ holds.

We introduce signatures and definitions separately to allow mutually recursive declarations. We do not check termination and also do not prevent re-definition.

\begin{mathpar}
\inferrule*[Right=Empty] { } {
  \checkDefJ \EMPTY
}
\end{mathpar}

\subsection{Functions}

\begin{mathpar}
\inferrule*[Right=Sig] {
  \checkDefJ \Sigma\\
    \checkJ \Sigma \EMPTY A \zero \Type\\
  (q \in \{\zero, \one\})
} {
  \checkDefJ {\Sigma, \Sig f q A}
}
\end{mathpar}

\begin{mathpar}
\inferrule*[Right=Def] {
  \checkDefJ \Sigma\\
  \left( \Sig f q A \right) \in \Sigma\\
  \checkJ \Sigma \EMPTY t q A\\
  \linCheckJ \EMPTY t q \emptyset\\
} {
  \checkDefJ {\Sigma, \Def f t}
}
\end{mathpar}
The last premise, $\linCheckJ \EMPTY t q \emptyset$, invokes the linearity checker.

\subsection{Data types}

\begin{mathpar}
\inferrule*[Right=Def-TyCon] {
  \checkJ \Sigma \EMPTY {\Delta \to \Type} \zero \Type
} {
  \checkDefJ
  {\Sigma,
    {\definition{tc} \OF {\Delta \to \Type}}
  }
}
\end{mathpar}

\begin{mathpar}
\inferrule*[Right=Def-DataCon] {
  \left( \definition{tc} \OF {\Delta \to \Type} \right) \in \Sigma\\\\
  \checkJ \Sigma \EMPTY {\Phi \to {\AppTele {\TyCon {tc}} \delta}} \zero \Type\\\\
  \textsf{length}(\delta) = \textsf{length}(\Delta)
} {
  \checkDefJ {\Sigma, {\definition{dc} \OF \Phi \to {\AppTele {\TyCon {tc}} \delta}}}
}
\end{mathpar}
Note: to keep the presentation simple, we leave data types open to extension. Strict-positivity check has not been implemented yet.

\section{Conversion}

Idris combines call-by-value evaluation with an untyped (and therefore incomplete) eta-conversion check\footnote{See \href{https://github.com/edwinb/Yaffle/blob/main/src/Core/Evaluate/Normalise.idr}{\texttt{Core.Evaluate.Normalise.evaluate}} and \href{https://github.com/edwinb/Yaffle/blob/main/src/Core/Evaluate/Convert.idr}{\texttt{Core.Evaluate.Convert.convGen}}.}. The plan is to make conversion checking typed.

The intended equational theory features the following computation rules
\begin{mathpar}
\inferrule*[Right=$\beta$] {
  \checkJ \Sigma {\restrictQ \Gamma p, (\name x \OFq q A)} e {\presence p} B\\
  \checkJ \Sigma \Gamma t {pq} A\\
} {
  \checkEqJ \Sigma \Gamma {\App q {\lam x q A e} t} {\Sub e t x} p {\Sub B t x}
}\\

\inferrule*[Right=$\eta$] {
  \checkJ \Sigma \Gamma f p {\PiT x q A B}\\
  \name x \not\in {\fv f}
} {
  \checkEqJ \Sigma \Gamma f {\lam x q A {\App q f {\name x}}} p {\PiT x q A B}
}\\

\inferrule*[Right=Let-$\delta$] {
  \checkJ \Sigma \Gamma A \zero \Type\\
  \checkJ \Sigma \Gamma t {pq} A\\
  \checkJ \Sigma {\restrictQ \Gamma p, (\name x \OFq q A)} s {\presence p} B
} {
  \checkEqJ \Sigma \Gamma {\Let x q A t s} {\Sub s t x} p B
}\\

\inferrule*[Right=Def-$\delta$] {
  \left( \Sig f q A \right) \in \Sigma\\
  \left( \Def f t \right) \in \Sigma\\
  \presence p \leq q
} {
  \checkEqJ \Sigma \Gamma {\definition f} t p A
}
\end{mathpar}
as well as the usual congruence rules.

The preliminary computation rule for case splitting looks as follows
\begin{mathpar}
\inferrule*[Right=Case-$\beta$]{
  \checkJ \Sigma \Gamma t {\multQE p q} A \\
  \checkPatJ \Sigma \Gamma A {l_i} {\Phi_i}\\
  \checkJ \Sigma {\restrictQ \Gamma p, \multQ {\Phi_i} q} {t_i} {\presence p} {?}\\\\
  \presence p = \one \land q = \zero \implies i \leq 1\\\\
  \matchJ t {\{\overline{\Clause {l_i} {t_i}}\}} {t'}
} {
  \checkEqJ \Sigma \Gamma {\Case t q A {\Clause {l_i} {t_i}}} {t'} q {?}
}
\end{mathpar}
where the $\textsf{Match}$ predicate encodes first-match semantics
\todo[noline]{Abuse of notation: need to turn environment into substitution.}\begin{mathpar}
\inferrule*[Right=Match-Default]{ } {
  \matchJ {t} {\{\Clause {\term {\_}} {t'}; {\overline{\Clause {l_i} {t_i}}}\}} {t'}
}\\

\inferrule*[Right=Match-Int]{ } {
  \matchJ {\term n} {\{\Clause {\term n} {t}; {\overline{\Clause {l_i} {t_i}}}\}} {t'}
}\\

\inferrule*[Right=Match-Int-Distinct]{
  \matchJ {\term n} {\overline{\Clause {l_i} {t_i}}} {t'}\\
  \term n \not= \term m
} {
  \matchJ {\term n} {\{\Clause {\term m} {t}; {\overline{\Clause {l_i} {t_i}}}\}} {t'}
}\\

\inferrule*[Right=Match-DataCon]{ } {
  \matchJ {\AppTele {\definition{dc}} \delta} {\{\Clause {\definition{dc}~ \Phi} {t}; {\overline{\Clause {l_i} {t_i}}}\}} {\SubE {t} \delta}
}\\

\inferrule*[Right=Match-DataCon-Distinct] {
  \matchJ {\AppTele {\definition{dc}} \delta} {\overline{\Clause {l_i} {t_i}}} {t'}\\
  \definition{dc} \not= \definition{dc'}
} {
  \matchJ {\AppTele {\definition{dc}} \delta} {\{\Clause {\definition{dc'}~ \Phi} {t}; {\overline{\Clause {l_i} {t_i}}}\}} {t'}
}
\end{mathpar}

\section{Linearity checking}\label{sec:linearity_checking}

We record usage of linear variables as a multiset $u \in \mathbb V \to \mathbb N$ with the following operations:

\begin{itemize}
\item empty usage
  $$
  \emptyset = \name y \mapsto 0
  $$
\item singleton usage
  $$
  \{ \name x \} = \name y \mapsto {\begin{cases}1 & \text{if $\name y = \name x$}\\ 0 & \text{otherwise} \end{cases}}
  $$
\item linear variable usage
  $$
  \used p {\name x} = {\begin{cases}\{\name x\} & \text{if $p = \one$}\\ \emptyset  & \text{otherwise} \end{cases}}
  $$
\item variable deletion
  $$
  u \setminus {\name x} = \name y \mapsto {\begin{cases}0 & \text{if $\name y = \name x$}\\ u(\name y) & \text{otherwise} \end{cases}}
  $$
\item union
  $$
  u \cup v = \name y \mapsto u(\name y) + v(\name y)
  $$
\end{itemize}

\subsection{Variables}

\begin{mathpar}
\inferrule*[Right=Lin-Var]{
  (\name x \OFq q A) \in \Gamma\\
  p \leq q
} {
  \linCheckJ \Gamma {\name x} p {\used p x}
}
\end{mathpar}

\subsection{Function type}
\begin{mathpar}
\inferrule*[Right=Lin-Pi]{
  \linCheckJ \Gamma A {\ceil p} {u_A} \\
  \linCheckJ {\Gamma, (\name x \OFq q {A})} B {\ceil p} {u_B}
} {
  \linCheckJ \Gamma {\PiT x q A B} p {u_A \cup ({u_B} \smallsetminus {\name x})}
}
\end{mathpar}
with
\begin{align*}
\ceil \zero &= \zero\\
\ceil \one &= \any\\
\ceil \any &= \any
\end{align*}
Note that $\ceil p = \multQ \any p$.

Intuitively, this rule forbids\todo[noline]{Ask Edwin for an explanation.} usage of linear variables inside function types that may be present at runtime.

\subsection{Function abstraction}

\begin{mathpar}
\inferrule*[Right=Lin-Lam]{
  \linCheckJ {\restrictQ \Gamma {p}, (\name x \OFq q A)} t {\presence p} u \\
  \presence p q = \one \implies u(\name x) = 1
} {
  \linCheckJ \Gamma {\lam x q A t} p {u \smallsetminus {\name x}}
}
\end{mathpar}

\subsection{Function application}
\begin{mathpar}
\inferrule*[Right=Lin-App] {
  \linCheckJ \Gamma f p {u_f} \\
  \linCheckJ \Gamma t {pq} {u_t}
} {
  \linCheckJ \Gamma {\App q f t} p {{u_f} \cup {u_t}}
}
\end{mathpar}

\subsection{Let binding}
\begin{mathpar}
\inferrule*[Right=Lin-Let]{
  \linCheckJ \Gamma t {\multQE p q} {u_t}\\
  \linCheckJ {\restrictQ \Gamma p, (\name x \OFq q A)} s {\presence p} {u_s}\\
  \presence p q = \one \implies u_s(\name x) = 1
} {
  \linCheckJ \Gamma {\Let x q A t s} p {u_t \cup (u_s \smallsetminus {\name x})}
}
\end{mathpar}

\subsection{Case}
\todo[noline]{Telescopic abuse of notation.}\begin{mathpar}
\inferrule*[Right=Lin-Case]{
  \linCheckJ \Gamma t {\multQE p q} {u_t} \\
  \checkPatJ \Sigma \Gamma A {l_i} {\Phi_i}\\
  \linCheckJ {\restrictQ \Gamma p, \multQ {\Phi_i} q} {t_i} {\presence p} {u_i}\\\\
  (\name x \OFq {q_i} A) \in \Phi_i \land \presence p q_i q = \one \implies u_i(\name x) = 1\\
  \presence p = \one \land q = \zero \implies i \leq 1
} {
  \linCheckJ \Gamma {\Case t q A {\Clause {l_i} {t_i}}} p {u_t \cup \bigcup_i {(u_i \smallsetminus \Phi_i)}}
}
\end{mathpar}

\subsection{Constants}
\begin{mathpar}
\inferrule*[Right=Lin-Int]{ } {
  \linCheckJ \Gamma {\type{Int}} p \emptyset
}
\end{mathpar}

\begin{mathpar}
\inferrule*[Right=Lin-Int-Const]{ } {
  \linCheckJ \Gamma {\term {n}} p \emptyset
}
\end{mathpar}

\subsection{Universe}
\begin{mathpar}
\inferrule*[Right=Lin-Type]{ } {
  \linCheckJ \Gamma \Type p \emptyset
}
\end{mathpar}

\printbibliography

\end{document}

%%% Local Variables:
%%% mode: latex
%%% TeX-master: t
%%% End:
