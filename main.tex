\documentclass{article}

\usepackage{amsmath}
\usepackage{amssymb}

\usepackage{mathtools}

\usepackage[utf8]{inputenc}
\usepackage{fontenc}

\usepackage{stmaryrd}

\usepackage{mathpartir}
\renewcommand{\TirNameStyle}[1]{{\scriptsize {\textsc {[#1]}}}}

\usepackage{url}
\usepackage{hyperref}

\usepackage{xcolor}

\definecolor{term}{RGB}{153,0,0}
\definecolor{type}{RGB}{0,0,224}
\definecolor{definition}{RGB}{0,102,0}
\definecolor{neutral}{RGB}{102,0,102}
\definecolor{meta}{RGB}{224,153,0}

\usepackage{todonotes}

%\usepackage[top=0.75in, bottom=1in, left=1in, right=1in]{geometry}

\title{Yaffle}

\newcommand{\used}[2]{\textsf{used}(#1,#2)}
\newcommand{\usageCount}[2]{\textsf{count}(#1,#2)}

\newcommand{\ceil}[1]{\left\lceil #1\right\rceil}
%\newcommand{\restrictOne}[1]{\left[ #1\right]_{\mathbb I}}
%\newcommand{\relevance}[1]{\left\vert#1\right\vert_{\zero\one}}
\newcommand{\relevance}[1]{\left\vert#1\right\vert}

\newcommand{\delete}[1]{\left\lvert #1\right\rvert}
\newcommand{\deleteQ}[2]{\textsf{erase}^{#2}\left(#1\right)}

\newcommand{\USES}{\triangleright}
\newcommand{\ENT}{\vdash}
\newcommand{\ENTq}[1]{\vdash^{#1}}
\newcommand{\OF}{:}
\newcommand{\TO}{\Rightarrow}
\newcommand{\OFq}[1]{\OF^{#1}}
\newcommand{\TOq}[1]{\TO^{#1}}
\newcommand{\EQ}{\ensuremath{\mathbin{\equiv}}}

\newcommand{\EMPTY}{.}

\newcommand{\checkJ}[4]{#1 \ENT #2 \OFq {#3} #4}
\newcommand{\inferJ}[4]{#1 \ENT #2 \TOq {#3} #4}

\newcommand{\type}[1]{{\textcolor{type}{\textsf{#1}}}}
\newcommand{\term}[1]{{\textcolor{term}{\textsf{#1}}}}
\newcommand{\definition}[1]{{\textcolor{definition}{\textsf{#1}}}}
\newcommand{\unv}[1]{{\textcolor{type}{\textsc{#1}}}}

\newcommand{\zero}{{\textcolor{type}{0}}}
\newcommand{\one}{{\textcolor{type}{1}}}
\newcommand{\any}{{\textcolor{type}{\omega}}}

\newcommand{\divQ}[2]{#1 \setminus #2}
\newcommand{\divAny}[1]{\divQ {#1} \any}

\newcommand{\name}[1]{{\textcolor{neutral}{#1}}}
\newcommand{\metaVariable}[1]{{\textcolor{meta}{#1}}}

\newcommand{\Type}{\unv{Type}}
\newcommand{\One}{\type{1}}
\newcommand{\To}{\mathbin{\textcolor{type}{\shortrightarrow}}}
% \newcommand{\App}[3]{#2^{#1}\, #3}
\newcommand{\App}[3]{\term{App}^{#1}\left(#2,#3\right)}
\newcommand{\Sub}[3]{#1 [#2/{\name #3}]}
\newcommand{\AppTele}[3]{\term{App}\left(#1\overline{,^{#2} {#3}}\right)}
\newcommand{\MetaApp}[3]{\term{MetaApp}\left(\metaVariable {#1} \overline{,^{#2} {#3}}\right)}
\newcommand{\TyCon}[2]{\type{TyCon}\left({\definition {#1}},#2\right)}
\newcommand{\DataCon}[2]{\term{DataCon}\left({\definition {#1}},#2\right)}
\newcommand{\lam}[1]{\textcolor{term}{\lambda} {\name #1}.\,}
\newcommand{\Let}[5]{\term{let}\, {\name #1} \OFq {#2} {#3} = {#4} \mathbin{\term{in}} {#5}}

\newcommand{\Def}[4]{\definition{#1} \OFq {#2} {#3} = {#4}}

\newcommand{\mult}{\mathbin{\textcolor{type}{\times}}}
\newcommand{\proj}[1]{\mathop{\textcolor{definition}{\pi_{#1}}}}
\newcommand{\cons}[2]{\textcolor{term}{[}#1\textcolor{term}{,}\, #2\textcolor{term}{]}}
\newcommand{\nil}{\textcolor{term}{[]}}

\newcommand{\SigmaT}[3]{\textcolor{type}{(}\name #1 \OF #2 \textcolor{type}{)} \mult #3}
\newcommand{\PiT}[4]{\textcolor{type}{(}\name #1 \OF^{#2} #3 \textcolor{type}{)} \To #4}

\newcommand{\PiTele}[4]{\overline{\textcolor{type}{(}\name #1 \OF^{#2} #3 \textcolor{type}{)}} \To #4}

\newcommand{\plus}{\mathbin{\textcolor{type}{+}}}
\newcommand{\inj}[1]{\mathop{\textcolor{term}{\iota_{#1}}}}
\newcommand{\match}[2]{\textcolor{definition}{\langle}#1\textcolor{definition}{,}\, #2\textcolor{definition}{\rangle}}

%%%%%%%%%%%%%%%%%%%%%%%%%%%%%%%%%%%%%%%%%%%%%%%%%%%%%%%%%%%%%%%%%%%%%%%%%%%%%%%%%%%%
% \newcommand{\cOne}{\con \App \cons \dOne \nil}                                   %
% \newcommand{\cSigma}[2]{\con \App \cons \dSigma {\cons {#1}  {\cons {#2} \nil}}} %
% \newcommand{\cInd}[1]{\con \App \cons \dInd {\cons #1 \nil}}                     %
% \newcommand{\cHind}[2]{\con \App \cons \dHind {\cons #1  {\cons #2 \nil}}}       %
%                                                                                  %
% \newcommand{\uOne}{\underline{\dOne}}                                            %
% \newcommand{\uSigma}[2]{\underline{\dSigma} \App #1 \App #2}                     %
% \newcommand{\uInd}[1]{\underline{\dInd} \App #1}                                 %
% \newcommand{\uHind}[2]{\underline{\dHind} \App #1 \App #2}                       %
%                                                                                  %
% \newcommand{\con}{\term{con}}                                                    %
%%%%%%%%%%%%%%%%%%%%%%%%%%%%%%%%%%%%%%%%%%%%%%%%%%%%%%%%%%%%%%%%%%%%%%%%%%%%%%%%%%%%

\usepackage{biblatex}
\bibliography{../references.bib}

\begin{document}

\maketitle

Idris features quantity annotations inspired by Atkey's QTT \cite{Atkey2018}. We extract a set of typing rules from Idris' linearity checking code for a hole-free segment of \textsf{Core}.

% \footnote{See \url{http://strictlypositive.org/dpm/colour.html} for the rationale behind the choice of colours.} to distinguish the different categories of syntax. In brief, \type{type formers} are blue, \term{constructors} red, \definition{defined functions} green, \name{bound variables} purple, and meta-variables stay uncoloured.

\begin{itemize}
\item quantities $p, q \mathrel{\text{::=}} \zero, \one, \any$ with $\zero < \one < \any$
\item contexts $\Gamma \mathrel{\text{::=}} \EMPTY \mid \Gamma, (\name x \OFq p T)$
\item variable usage $u$ is a multiset of variables
\end{itemize}

Let

\begin{mathpar}
\used p x = \begin{cases}
\{x\} & \text{if $p = 1$}\\
\emptyset  & \text{otherwise}
\end{cases}
\end{mathpar}

\begin{mathpar}
\begin{array}{l}
\ceil \zero = \zero\\
\ceil \one = \omega\\
\ceil \omega = \omega
\end{array}
\end{mathpar}

Note that $\ceil p = \any \cdot p$

\begin{mathpar}
\begin{array}{l}
\divAny \zero = \zero\\
\divAny \one = \zero\\
\divAny \any = \any
\end{array}\\
  
\begin{array}{ll}
\divAny {\EMPTY} = \EMPTY\\
\divAny {\Gamma, (x \OFq p T)} = \divAny \Gamma, (x \OFq {\divAny p} T)
\end{array}
\end{mathpar}

\begin{mathpar}
\begin{array}{l}
\relevance \zero = \zero\\
\relevance \one = \one\\
\relevance \any = \one
\end{array}
\end{mathpar}

\section{Quantity-annotated judgements}

Bidirectional discipline. Judgement $\checkJ \Gamma t p A$ amounts to checking that term $t$ has type $A$ in context $\Gamma$ at {\em ambient} quantity $p$. We write the corresponding type inference judgement as $\inferJ \Gamma t p A$.

optimisation

\section{Term: \texttt{linearCheck}}

Judgment $\Gamma \ENTq p t \leadsto {t'} \USES u$:

%%%%%%%%%%%%%%%%%%%%%%%%%%%%%%%%%%%%%%%%%%%%%%%%%%%%%%%%%%%%%%%%%%%%%%%%%%%%%%%
% The erasure flag $e$ (\texttt{erase}) switches between two modes            %
%                                                                             %
% \begin{itemize}                                                             %
% \item {\keep} (\texttt{erase = false}): linearity check, erasure disabled   %
% \item {\erase} (\texttt{erase = true}): linearity disabled, erasure enabled %
% \end{itemize}                                                               %
%%%%%%%%%%%%%%%%%%%%%%%%%%%%%%%%%%%%%%%%%%%%%%%%%%%%%%%%%%%%%%%%%%%%%%%%%%%%%%%


\begin{mathpar}
\inferrule*[Right=term]{
  \Gamma \ENTq p t \leadsto {t'} \USES u\\
  \Gamma \ENTq p u
} {
  \Gamma \ENTq p t \leadsto {t'}
}
\end{mathpar}

\section{Context: \texttt{checkEnvUsage}}
\begin{mathpar}
\inferrule*[Right=ctx-{[]}]{ } {
  \EMPTY \ENTq p u
}\\

\inferrule*[Right=ctx-{::}]{
  \Gamma \ENTq p u\\
  p q = \one \implies u(x) = 1
} {
  \Gamma, (x \OFq q T) \ENTq p u
}

\end{mathpar}

\section{Subterms: \texttt{lcheck}, \texttt{lcheckBinder}}

\subsection{Variables}

\begin{mathpar}
\inferrule*[Right=Var]{
  (\name x \OFq q A) \in \Gamma\\
  p \leq q
} {
  \inferJ \Gamma {\name x} q A
}
\end{mathpar}

\subsection{Function type}

\begin{mathpar}
\inferrule*[Right=Pi]{
  \Gamma \ENTq {\ceil p q} A \leadsto {A'} \USES \_ \\\\
  \Gamma, (\name x \OFq q {A'}) \ENTq {\ceil p} B \leadsto {B'} \USES u
} {
  \Gamma \ENTq p \PiT x q A B \leadsto \PiT x q {A'} {B'} \USES u \smallsetminus \name x
}
\end{mathpar}
Note: when $p$ is runtime-relevant, $B$ can only mention argument $x$ if it has quantity $\any$.

\begin{mathpar}
\inferrule*[Right=Pi-hack]{
  \Gamma \not\ENTq {\ceil p q} A \leadsto \_ \USES \_ \\
  \Gamma \ENTq {\ceil p \one} A \leadsto {A'} \USES \_ \\
  (p \not= \zero \land q = \zero)\\\\
  \Gamma, (\name x \OFq \one {A'}) \ENTq {\ceil p} B \leadsto {B'} \USES u
} {
  \Gamma \ENTq p \PiT x q A B \leadsto \PiT x \one {A'} {B'} \USES u \smallsetminus \name x
}
\end{mathpar}

\textsc{Pi-hack} has contradictory premises if $p = \zero$ or $q \not= \zero$:
\begin{mathpar}
\begin{array}{ cc|cc }
p & q & \ceil p q & \ceil p \one \\
\hline
\one & \zero & \zero & \any \\
\any & \zero & \zero & \any
\end{array}
\end{mathpar}
Furthermore, $q = \zero$ forces $\ceil p q = \zero$, meaning whenever $\Gamma \ENTq {\ceil p \one} A \leadsto A' \USES u$, we can also derive $\Gamma \ENTq {\ceil p q} A \leadsto \_ \USES \_$. Hence, the rule never applies.

\subsection{Function abstraction}

\begin{mathpar}
\inferrule*[Right=Lam]{
  \checkJ {\divQ \Gamma p, (\name x \OFq q A)} t {\relevance p} B
} {
  \checkJ \Gamma {\lam x t} p {\PiT x q A B}
}
\end{mathpar}

\subsection{Function application}
\begin{mathpar}
\inferrule*[Right=App] {
  \inferJ \Gamma f p {\PiT x q A B} \\
  \checkJ \Gamma t {pq} A
} {
  \inferJ \Gamma {\App q f t} p {\Sub B t x}
}
\end{mathpar}

Note: no usage annotation on application.

\subsection{Let binding}
\begin{mathpar}
\inferrule*[Right=Let]{
  \Gamma \ENTq \zero A \leadsto {A'} \USES \_ \\
  \Gamma \ENTq {p q} t \leadsto {t'} \USES u_t \\
  \Gamma, (\name x \OFq q {A'}) \ENTq p s \leadsto {s'} \USES u_s \\
  p q = \one \implies u_s(\name x) = 1
} {
  \Gamma \ENTq p \Let x q A t s \leadsto \Let x q {A'} {t'} {s'} \USES u_t \cup (u_s \smallsetminus \name x)
}
\end{mathpar}
This rule is unnecessarily restrictive: we are not allowed to type a let-binding when $p = \any$ and $q = \one$ even if $\Gamma \ENTq \any t \leadsto {t'} \USES u_t$ is derivable (c.f. \TirName{Lam-$\any$}).

\begin{mathpar}
\inferrule*[Right=Let-hack]{
  \Gamma \ENTq \zero A \leadsto {A'} \USES \_ \\
  \Gamma \not\ENTq {p q} t \leadsto \_ \USES \_ \\
  \Gamma \ENTq {p \one} t \leadsto {t'} \USES u_t \\
% (p \not= 0 \land q \not= \one \land (p = \any \Leftrightarrow q = \zero))\\\\
  (p \not= 0 \land q \not= \one \land p \not= q)\\\\
  \Gamma, (\name x \OFq \one {A'}) \ENTq p s \leadsto {s'} \USES u_s
} {
  \Gamma \ENTq p \Let x q A t s \leadsto \Let x \one {A'} {t'} {s'} \USES u_t \cup (u_s \smallsetminus \name x)
}
\end{mathpar}
Non-contradictory choices for $p$ and $q$:
\begin{mathpar}
\begin{array}{ cc|cc }
p & q & p q & p \one \\
\hline
\one & \zero & \zero & \one \\
\one & \any  & \any  & \one \\
\any & \zero & \zero & \any
\end{array}
\end{mathpar}
If $q = \zero$ and $\Gamma \ENTq {p \one} t \leadsto t' \USES u_t$, then necessarily\todo{Subject to proof!} $\Gamma \ENTq {p q} t \leadsto \_ \USES \_$ (since $p q = \zero$). So the rule can only apply when $p = \one$ and $q = \any$, i.e.

\begin{mathpar}
\inferrule*[Right=Let-hack-$\one\any$]{
  \Gamma \ENTq \zero A \leadsto {A'} \USES \_ \\
  \Gamma \not\ENTq \any t \leadsto \_ \USES \_ \\
  \Gamma \ENTq \one t \leadsto {t'} \USES u_t \\
  \Gamma, (\name x \OFq \one {A'}) \ENTq \one s \leadsto {s'} \USES u_s
} {
  \Gamma \ENTq \one \Let x \any A t s \leadsto \Let x \one {A'} {t'} {s'} \USES u_t \cup (u_s \smallsetminus \name x)
}
\end{mathpar}


Note: we do not erase unused let-bound terms.

\subsection{Meta-variables}
\begin{mathpar}
\inferrule*[Right=MetaApp-def] {
  (X = d) \in \Sigma \\
  \Gamma \ENTq {pq_i} t_i \leadsto {t'_i} \USES u_i
} {
  \Gamma \ENTq p {\MetaApp X {q_i} {t_i}} \leadsto \MetaApp X {q_i} {\deleteQ {t'_i} {q_i}} \USES \bigcup u_i
}

\inferrule*[Right=MetaApp-undef] {
  X \in \Sigma
} {
  \Gamma \ENTq p {\MetaApp X {q_i} {t_i}} \leadsto \MetaApp X {q_i} {\deleteQ {t_i} {q_i}} \USES \emptyset
}
\end{mathpar}

\subsection{Data types}

% TODO check types
\begin{mathpar}
\inferrule*[Right=TyCon] {
  \definition{tc} \OFq q {\PiTele {x_i} {q_i} {T_i} \Type} \in \Sigma \\
  p \leq q
} {
  \Gamma \ENTq p \TyCon {tc} n \leadsto \TyCon {tc} n \USES \emptyset
}\\

\inferrule*[Right=DataCon] {
  \definition{dc} \OFq q {\PiTele {y_i} {r_i} {A_i} {\AppTele {\TyCon {tc} n} {q_i} {t_i}}} \in \Sigma \\
  p \leq q
} {
  \Gamma \ENTq p \DataCon {dc} n \leadsto \DataCon {dc} n \USES \emptyset
}
\end{mathpar}

\subsection{Pattern matching}

\subsection{Definition}
\begin{mathpar}
\inferrule*[Right=ref] {
  \definition f \OFq q T \in \Sigma\\
  p \leq q
} {
  \Gamma \ENTq p \definition f \leadsto \definition f \USES \emptyset
}
\end{mathpar}


\begin{mathpar}
\inferrule*[Right=def] {
  \EMPTY \ENTq \zero T \leadsto {T'} \USES \_\\
  \EMPTY \ENTq q t \leadsto {t'} \USES u
} {
  {} \ENT {\Def f q T t} \leadsto {\Def f q {T'} {t'}}
}
\end{mathpar}

\printbibliography

\end{document}

%%% Local Variables:
%%% mode: latex
%%% TeX-master: t
%%% End:
