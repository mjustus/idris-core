\documentclass{article}

\usepackage{amsmath}
\usepackage{amssymb}

\usepackage{mathtools}

\usepackage[utf8]{inputenc}
\usepackage{fontenc}

\usepackage{stmaryrd}

\usepackage{mathpartir}
\renewcommand{\TirNameStyle}[1]{{\scriptsize {\textsc {[#1]}}}}

\usepackage{url}
\usepackage{hyperref}

\usepackage{xcolor}

\definecolor{term}{RGB}{153,0,0}
\definecolor{type}{RGB}{0,0,224}
\definecolor{definition}{RGB}{0,102,0}
\definecolor{neutral}{RGB}{102,0,102}
\definecolor{meta}{RGB}{224,153,0}

\usepackage{todonotes}

%\usepackage[top=0.75in, bottom=1in, left=1in, right=1in]{geometry}

\title{Yaffle}

\newcommand{\used}[2]{\textsf{used}(#1,#2)}
\newcommand{\usageCount}[2]{\textsf{count}(#1,#2)}

\newcommand{\ceil}[1]{\left\lceil #1\right\rceil}
%\newcommand{\restrictOne}[1]{\left[ #1\right]_{\mathbb I}}
%\newcommand{\presence}[1]{\left\vert#1\right\vert_{\zero\one}}
\newcommand{\presence}[1]{\left\vert#1\right\vert}

\newcommand{\delete}[1]{\left\lvert #1\right\rvert}
\newcommand{\deleteQ}[2]{\textsf{erase}^{#2}\left(#1\right)}

\newcommand{\USES}{\triangleright}
\newcommand{\ENT}{\vdash}
\newcommand{\ENTq}[1]{\vdash^{#1}}
\newcommand{\OF}{:}
\newcommand{\TO}{\Rightarrow}
\newcommand{\INq}[1]{}
\newcommand{\OFq}[1]{\OF^{#1}}
\newcommand{\TOq}[1]{\TO^{#1}}
\newcommand{\EQ}{\ensuremath{\mathbin{\equiv}}}

\newcommand{\EMPTY}{.}

\newcommand{\checkJ}[4]{#1 \ENT {#2} \in^{#3} #4}
\newcommand{\inferJ}[4]{#1 \ENT #2 \TOq {#3} #4}

\newcommand{\type}[1]{{\textcolor{type}{\textsf{#1}}}}
\newcommand{\term}[1]{{\textcolor{term}{\textsf{#1}}}}
\newcommand{\definition}[1]{{\textcolor{definition}{\textsf{#1}}}}
\newcommand{\unv}[1]{{\textcolor{type}{\textsc{#1}}}}

\newcommand{\zero}{{\textcolor{type}{0}}}
\newcommand{\one}{{\textcolor{type}{1}}}
\newcommand{\any}{{\textcolor{type}{\omega}}}

\newcommand{\restrictQ}[2]{#1 \setminus #2}
\newcommand{\multQ}[2]{{#1} \cdot {#2}}
\newcommand{\restrictAny}[1]{\restrictQ {#1} \any}

\newcommand{\name}[1]{{\textcolor{neutral}{#1}}}
\newcommand{\metaVariable}[1]{{\textcolor{meta}{#1}}}

\newcommand{\Type}{\unv{Type}}
\newcommand{\One}{\type{1}}
\newcommand{\To}{\mathbin{\textcolor{type}{\shortrightarrow}}}
% \newcommand{\App}[3]{#2^{#1}\, #3}
\newcommand{\App}[3]{\term{App}^{#1}\left(#2,#3\right)}
\newcommand{\Sub}[3]{#1 [#2/{\name #3}]}
\newcommand{\AppTele}[3]{\term{App}\left(#1\overline{,^{#2} {#3}}\right)}
\newcommand{\MetaApp}[3]{\term{MetaApp}\left(\metaVariable {#1} \overline{,^{#2} {#3}}\right)}
\newcommand{\TyCon}[2]{\type{TyCon}\left({\definition {#1}},#2\right)}
\newcommand{\DataCon}[2]{\term{DataCon}\left({\definition {#1}},#2\right)}
\newcommand{\lam}[1]{\textcolor{term}{\lambda} {\name #1}.\,}
\newcommand{\Let}[5]{\term{let}\, {\name #1} \OFq {#2} {#3} = {#4} \mathbin{\term{in}} {#5}}

\newcommand{\Def}[4]{\definition{#1} \OFq {#2} {#3} = {#4}}

\newcommand{\mult}{\mathbin{\textcolor{type}{\times}}}
\newcommand{\proj}[1]{\mathop{\textcolor{definition}{\pi_{#1}}}}
\newcommand{\cons}[2]{\textcolor{term}{[}#1\textcolor{term}{,}\, #2\textcolor{term}{]}}
\newcommand{\nil}{\textcolor{term}{[]}}

\newcommand{\SigmaT}[3]{\textcolor{type}{(}\name #1 \OF #2 \textcolor{type}{)} \mult #3}
\newcommand{\PiT}[4]{\textcolor{type}{(}\name #1 \OF^{#2} #3 \textcolor{type}{)} \To #4}

\newcommand{\PiTele}[4]{\overline{\textcolor{type}{(}\name #1 \OF^{#2} #3 \textcolor{type}{)}} \To #4}

\newcommand{\plus}{\mathbin{\textcolor{type}{+}}}
\newcommand{\inj}[1]{\mathop{\textcolor{term}{\iota_{#1}}}}
\newcommand{\match}[2]{\textcolor{definition}{\langle}#1\textcolor{definition}{,}\, #2\textcolor{definition}{\rangle}}

%%%%%%%%%%%%%%%%%%%%%%%%%%%%%%%%%%%%%%%%%%%%%%%%%%%%%%%%%%%%%%%%%%%%%%%%%%%%%%%%%%%%
% \newcommand{\cOne}{\con \App \cons \dOne \nil}                                   %
% \newcommand{\cSigma}[2]{\con \App \cons \dSigma {\cons {#1}  {\cons {#2} \nil}}} %
% \newcommand{\cInd}[1]{\con \App \cons \dInd {\cons #1 \nil}}                     %
% \newcommand{\cHind}[2]{\con \App \cons \dHind {\cons #1  {\cons #2 \nil}}}       %
%                                                                                  %
% \newcommand{\uOne}{\underline{\dOne}}                                            %
% \newcommand{\uSigma}[2]{\underline{\dSigma} \App #1 \App #2}                     %
% \newcommand{\uInd}[1]{\underline{\dInd} \App #1}                                 %
% \newcommand{\uHind}[2]{\underline{\dHind} \App #1 \App #2}                       %
%                                                                                  %
% \newcommand{\con}{\term{con}}                                                    %
%%%%%%%%%%%%%%%%%%%%%%%%%%%%%%%%%%%%%%%%%%%%%%%%%%%%%%%%%%%%%%%%%%%%%%%%%%%%%%%%%%%%

\usepackage{biblatex}
\bibliography{references.bib}

\begin{document}

\maketitle

We extract a set of typing rules for the hole-free fragment of Idris' new core, Yaffle\footnote{See \href{https://github.com/edwinb/Yaffle/blob/main/src/Core/Check/Typecheck.idr}{{\texttt{Core.Check.Typecheck}}} and \href{https://github.com/edwinb/Yaffle/blob/main/src/Core/Check/Linear.idr}{\texttt{Core.Check.Linear}} in repository \url{https://github.com/edwinb/Yaffle}.}

Idris splits the world into expressions that are present at runtime and those that may be erased after type-checking. This is enforced via quantity annotations on variables in context
\begin{itemize}
\item quantities $p, q \mathrel{\text{::=}} \zero, \one, \any$ with $\zero < \one < \any$
\item contexts $\Gamma \mathrel{\text{::=}} \EMPTY \mid \Gamma, (\name x \OFq q T)$
\end{itemize}
Informally, runtime-present expressions can only depend on variables at quantity $\one$ or $\any$, where $\one$-annotated variables must be used exactly once and $\any$-annotated variables can be used arbitrarily. In contrast, erased expressions have unrestricted access to all variables, including $\zero$-annotated variables.

The context records the number of copies available of each variable; it does not account for exact variable usage inside the term to be checked\footnote{Unlike Atkey's QTT\cite{Atkey2018}.}. Type checking only enforces that variables usage adheres to this budget locally but, in particular, does not enforce that linear variables are used precisely once. This is left to the \hyperref[sec:linearity_checking]{linearity checker}, which runs after type checking.

We will frequently need to multiply quantities, turning the set of  quantities into a monoid
\begin{mathpar}
\begin{array}{l}
\multQ \one y = y\\
\multQ x \one = x\\\\
\multQ \zero y = \zero\\
\multQ x \zero = \zero\\\\
\multQ \any \any = \any
\end{array}
\end{mathpar}

When manipulating the context, we will also need a restriction operation $\restrictQ p q$, which makes $p$-annotated variables that do not support at least $q$ copies unavailable at runtime
\begin{mathpar}
\restrictQ p q =
  \left\{
    \begin{array}{ll}
      p & \text{if $p \leq q$} \\
      \zero & \text{otherwise}
    \end{array}
  \right.
\end{mathpar}
Concretely,
\begin{mathpar}
\begin{array}{l}
\restrictQ p \zero = p\\
\restrictQ p \one = p\\
\restrictAny \zero = \zero\\
\restrictAny \one = \zero\\
\restrictAny \any = \any
\end{array}
\end{mathpar}
and lifted to contexts
\begin{mathpar}
\begin{array}{ll}
\restrictQ {\EMPTY} p = \EMPTY\\
\restrictQ {\Gamma, (x \OFq q T)} p = \restrictQ \Gamma p, (x \OFq {\restrictQ q p} T)
\end{array}
\end{mathpar}
Note that $\Gamma = \restrictQ \Gamma \zero = \restrictQ \Gamma \one$.

\section{Quantity-annotated judgements}

The type-checker follows a bidirectional discipline. Judgement $\checkJ \Gamma t p A$ amounts to checking that term $t$ has type $A$ in context $\Gamma$ at {\em ambient} quantity $p$. The corresponding inference judgement will be written as $\inferJ \Gamma t p A$.

The ambient quantity has dual purpose. Similar to QTT's erased/present flag\cite{Atkey2018}, it controls whether the expression to be checked is considered to be erased ($\zero$) or present ($\one$ or $\any$). It also acts as cost multiplier for variable accesses: suppose we want to infer the type of an application $\App \any f t$ at ambient quantity $\one$ and already know that $\inferJ \Gamma f \one {\PiT x \any A B}$. We now not only need to check that argument $t$ has type $A$ but also that we are allowed to make $\any$ copies of $t$. One way to achieve this would be to restrict the context to those variables that support $\any$ copies, i.e. $\checkJ {\restrictAny \Gamma} t {\one} A$. Idris, however, uses a variable rule that requires any accessed variable to have quantity larger or equal to ambient quanitity
\begin{mathpar}
\inferrule*[Right=Var]{
  (\name x \OFq q A) \in \Gamma\\
  p \leq q
} {
  \inferJ \Gamma {\name x} p A
}
\end{mathpar}
allowing us to use $\checkJ \Gamma t \any A$ instead. In fact, we should always have $$\checkJ \Gamma t {p} A \iff \checkJ {\restrictQ \Gamma p} t {\presence p} A$$ where $\presence p$ squashes quantity $p$ down to just its presence
\begin{mathpar}
\begin{array}{l}
\presence \zero = \zero\\
\presence \one = \one\\
\presence \any = \one
\end{array}
\end{mathpar}


%%%%%%%%%%%%%%%%%%%%%%%%%%%%%%%%%%%%%%
% variable rule would be             %
%                                    %
% \begin{mathpar}                    %
% \inferrule*[Right=Var-$\zero$]{    %
%   (\name x \OFq q A) \in \Gamma    %
% } {                                %
%   \inferJ \Gamma {\name x} \zero A %
% }                                  %
% \\                                 %
% \inferrule*[Right=Var-$\one$]{     %
%   (\name x \OFq q A) \in \Gamma\\  %
%   q \not= \zero                    %
% } {                                %
%   \inferJ \Gamma {\name x} \one A  %
% }                                  %
% \end{mathpar}                      %
%%%%%%%%%%%%%%%%%%%%%%%%%%%%%%%%%%%%%%

\section{Type checking}}

\subsection{Variables}

\begin{mathpar}
\inferrule*[Right=Var]{
  (\name x \OFq q A) \in \Gamma\\
  p \leq q
} {
  \inferJ \Gamma {\name x} p A
}
\end{mathpar}

\subsection{Function type}

\begin{mathpar}
\inferrule*[Right=Pi]{
  \Gamma \ENTq {\ceil p q} A \leadsto {A'} \USES \_ \\\\
  \Gamma, (\name x \OFq q {A'}) \ENTq {\ceil p} B \leadsto {B'} \USES u
} {
  \Gamma \ENTq p \PiT x q A B \leadsto \PiT x q {A'} {B'} \USES u \smallsetminus \name x
}
\end{mathpar}
Note: when $p$ is runtime-present, $B$ can only mention argument $x$ if it has quantity $\any$.

\begin{mathpar}
\inferrule*[Right=Pi-hack]{
  \Gamma \not\ENTq {\ceil p q} A \leadsto \_ \USES \_ \\
  \Gamma \ENTq {\ceil p \one} A \leadsto {A'} \USES \_ \\
  (p \not= \zero \land q = \zero)\\\\
  \Gamma, (\name x \OFq \one {A'}) \ENTq {\ceil p} B \leadsto {B'} \USES u
} {
  \Gamma \ENTq p \PiT x q A B \leadsto \PiT x \one {A'} {B'} \USES u \smallsetminus \name x
}
\end{mathpar}

\textsc{Pi-hack} has contradictory premises if $p = \zero$ or $q \not= \zero$:
\begin{mathpar}
\begin{array}{ cc|cc }
p & q & \ceil p q & \ceil p \one \\
\hline
\one & \zero & \zero & \any \\
\any & \zero & \zero & \any
\end{array}
\end{mathpar}
Furthermore, $q = \zero$ forces $\ceil p q = \zero$, meaning whenever $\Gamma \ENTq {\ceil p \one} A \leadsto A' \USES u$, we can also derive $\Gamma \ENTq {\ceil p q} A \leadsto \_ \USES \_$. Hence, the rule never applies.

\subsection{Function abstraction}

\begin{mathpar}
\inferrule*[Right=Lam]{
  \checkJ {\restrictQ \Gamma p, (\name x \OFq q A)} t {\presence p} B
} {
  \checkJ \Gamma {\lam x t} p {\PiT x q A B}
}
\end{mathpar}

\subsection{Function application}
\begin{mathpar}
\inferrule*[Right=App] {
  \inferJ \Gamma f p {\PiT x q A B} \\
  \checkJ \Gamma t {pq} A
} {
  \inferJ \Gamma {\App q f t} p {\Sub B t x}
}
\end{mathpar}

further optimisation: only restrict the context when introduction a linear variable to the context at ambient quantity $\any$.
not replicate in the rules

\subsection{Let binding}
\begin{mathpar}
\inferrule*[Right=Let]{
  \Gamma \ENTq \zero A \leadsto {A'} \USES \_ \\
  \Gamma \ENTq {p q} t \leadsto {t'} \USES u_t \\
  \Gamma, (\name x \OFq q {A'}) \ENTq p s \leadsto {s'} \USES u_s \\
  p q = \one \implies u_s(\name x) = 1
} {
  \Gamma \ENTq p \Let x q A t s \leadsto \Let x q {A'} {t'} {s'} \USES u_t \cup (u_s \smallsetminus \name x)
}
\end{mathpar}
This rule is unnecessarily restrictive: we are not allowed to type a let-binding when $p = \any$ and $q = \one$ even if $\Gamma \ENTq \any t \leadsto {t'} \USES u_t$ is derivable (c.f. \TirName{Lam-$\any$}).

\begin{mathpar}
\inferrule*[Right=Let-hack]{
  \Gamma \ENTq \zero A \leadsto {A'} \USES \_ \\
  \Gamma \not\ENTq {p q} t \leadsto \_ \USES \_ \\
  \Gamma \ENTq {p \one} t \leadsto {t'} \USES u_t \\
% (p \not= 0 \land q \not= \one \land (p = \any \Leftrightarrow q = \zero))\\\\
  (p \not= 0 \land q \not= \one \land p \not= q)\\\\
  \Gamma, (\name x \OFq \one {A'}) \ENTq p s \leadsto {s'} \USES u_s
} {
  \Gamma \ENTq p \Let x q A t s \leadsto \Let x \one {A'} {t'} {s'} \USES u_t \cup (u_s \smallsetminus \name x)
}
\end{mathpar}
Non-contradictory choices for $p$ and $q$:
\begin{mathpar}
\begin{array}{ cc|cc }
p & q & p q & p \one \\
\hline
\one & \zero & \zero & \one \\
\one & \any  & \any  & \one \\
\any & \zero & \zero & \any
\end{array}
\end{mathpar}
If $q = \zero$ and $\Gamma \ENTq {p \one} t \leadsto t' \USES u_t$, then necessarily\todo{Subject to proof!} $\Gamma \ENTq {p q} t \leadsto \_ \USES \_$ (since $p q = \zero$). So the rule can only apply when $p = \one$ and $q = \any$, i.e.

\begin{mathpar}
\inferrule*[Right=Let-hack-$\one\any$]{
  \Gamma \ENTq \zero A \leadsto {A'} \USES \_ \\
  \Gamma \not\ENTq \any t \leadsto \_ \USES \_ \\
  \Gamma \ENTq \one t \leadsto {t'} \USES u_t \\
  \Gamma, (\name x \OFq \one {A'}) \ENTq \one s \leadsto {s'} \USES u_s
} {
  \Gamma \ENTq \one \Let x \any A t s \leadsto \Let x \one {A'} {t'} {s'} \USES u_t \cup (u_s \smallsetminus \name x)
}
\end{mathpar}


Note: we do not erase unused let-bound terms.

\subsection{Meta-variables}
\begin{mathpar}
\inferrule*[Right=MetaApp-def] {
  (X = d) \in \Sigma \\
  \Gamma \ENTq {pq_i} t_i \leadsto {t'_i} \USES u_i
} {
  \Gamma \ENTq p {\MetaApp X {q_i} {t_i}} \leadsto \MetaApp X {q_i} {\deleteQ {t'_i} {q_i}} \USES \bigcup u_i
}

\inferrule*[Right=MetaApp-undef] {
  X \in \Sigma
} {
  \Gamma \ENTq p {\MetaApp X {q_i} {t_i}} \leadsto \MetaApp X {q_i} {\deleteQ {t_i} {q_i}} \USES \emptyset
}
\end{mathpar}

\subsection{Data types}

% TODO check types
\begin{mathpar}
\inferrule*[Right=TyCon] {
  \definition{tc} \OFq q {\PiTele {x_i} {q_i} {T_i} \Type} \in \Sigma \\
  p \leq q
} {
  \Gamma \ENTq p \TyCon {tc} n \leadsto \TyCon {tc} n \USES \emptyset
}\\

\inferrule*[Right=DataCon] {
  \definition{dc} \OFq q {\PiTele {y_i} {r_i} {A_i} {\AppTele {\TyCon {tc} n} {q_i} {t_i}}} \in \Sigma \\
  p \leq q
} {
  \Gamma \ENTq p \DataCon {dc} n \leadsto \DataCon {dc} n \USES \emptyset
}
\end{mathpar}

\subsection{Pattern matching}

\subsection{Definition}
\begin{mathpar}
\inferrule*[Right=ref] {
  \definition f \OFq q T \in \Sigma\\
  p \leq q
} {
  \Gamma \ENTq p \definition f \leadsto \definition f \USES \emptyset
}
\end{mathpar}


\begin{mathpar}
\inferrule*[Right=def] {
  \EMPTY \ENTq \zero T \leadsto {T'} \USES \_\\
  \EMPTY \ENTq q t \leadsto {t'} \USES u
} {
  {} \ENT {\Def f q T t} \leadsto {\Def f q {T'} {t'}}
}
\end{mathpar}

\section{Linearity checking}\label{sec:linearity_checking}


\begin{itemize}
\item variable usage $u$ is a multiset of variables
\end{itemize}

Let

\begin{mathpar}
\used p x = \begin{cases}
\{x\} & \text{if $p = 1$}\\
\emptyset  & \text{otherwise}
\end{cases}
\end{mathpar}


\printbibliography

\end{document}

%%% Local Variables:
%%% mode: latex
%%% TeX-master: t
%%% End:
